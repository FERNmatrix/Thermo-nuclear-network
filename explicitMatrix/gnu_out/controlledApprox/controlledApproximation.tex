
\documentclass[11pt]{article}

\usepackage{graphicx}% Include figure files

% Following package must be loaded before amsthm.sty
\usepackage{amsmath}       
\usepackage{amssymb}  % Use instead of latexsym to get boxes, etc
\usepackage{amsthm} 

% Following to replace computer modern with times font
\usepackage{mathptmx, courier, pifont}
\usepackage[scaled=0.92]{helvet}
\usepackage[T1]{fontenc}
\usepackage{textcomp}

\usepackage{bm}% bold math

%%  \putfig usage:
%%  \putfig
%%      {label}
%%      {file.eps}
%%      {scale factor, 1.0 is no scaling}
%%      {caption}

\newcommand{\putfig}[6]{%
\begin{figure}%
\vspace*{#2}
\centering
\includegraphics[scale=#5]{#4}%
\caption{#6}%
\label{fig:#1}%
\vspace*{#3}
\end{figure}
}

% symbol definitions:
\newcommand{\tsub}[1]{_{\mbox{\scriptsize#1}}}
\newcommand{\tsup}[1]{^{\mbox{\scriptsize#1}}}
\newcommand{\ttsub}[1]{_{\mbox{\tiny#1}}}
\newcommand{\ttsup}[1]{^{\mbox{\tiny#1}}}
\newcommand{\quarterthin}{\kern 0.0417em}
\newcommand{\comm}[2]{[ \quarterthin #1 , #2 \quarterthin ]}
\newcommand{\bra}[1]{\langle#1|}
\newcommand{\ket}[1]{|#1\rangle}
\newcommand{\ev}[1]{\langle#1\rangle}
\newcommand{\mel}[3]{\bra{#1}#2\ket{#3}}
\newcommand{\thin}{\thinspace}
\newcommand{\halfthin}{\kern 0.0834em}
\newcommand{\neghalfthin}{\kern -0.0834em}
\newcommand{\negquarterthin}{\kern - 0.0417em}
\newcommand{\diffelement}[1]{d\negquarterthin#1}
\newcommand{\deriv}[2]{\frac{\diffelement#1}{\diffelement#2}}
\newcommand{\pardiv}[2]{\frac{\partial {\negquarterthin #1}}
{\partial {\negquarterthin #2}}}
\newcommand\gcm{g~cm$^{-3}$}
\newcommand\simgreater{\,\lower0.7ex\hbox{$\stackrel{>}{\sim}$}\,}
\newcommand\simless{\,\lower0.7ex\hbox{$\stackrel{<}{\sim}$}\,}
\newcommand{\units}[1]{\mbox{\ #1}}

\newcommand{\Fplus}[1]{F^+_{#1}}
\newcommand{\Fminus}[1]{F^-_{#1}}
\newcommand{\fplus}[1]{f^+_{#1}}
\newcommand{\fminus}[1]{f^-_{#1}}
\newcommand{\PEsource}[2]{f_{{#1}{\rightleftharpoons}{#2}}}
\newcommand{\eq}[1]{Eq.~(\ref{#1})}
\newcommand{\eqnoeq}[1]{(\ref{#1})}
\newcommand{\fig}[1]{Fig.~\ref{fig:#1}}
\newcommand{\tableref}[1]{Table \ref{tb:#1}}
\newcommand{\secref}[1]{Section~\ref{#1}}

\newcommand{\isotope}[2]{\mbox{$^{#1}$#2}}
\newcommand{\reaction}[8]{\isotope{#1}{#2}(\isotope{#3}{#4},\isotope{#5}{#6})
                          \isotope{#7}{#8}}

\newcommand{\kforward}[1]{k^{(#1)}_{\rm\scriptstyle f}}
\newcommand{\kback}[1]{k^{(#1)}_{\rm\scriptstyle r}}

% In order to get figures larger than 4 inches tall on a page 
% and still have some text printed on the page, and to get less blank
% space between the figure caption and the text below, you should add the 
% following (BEFORE \begin{document}):

\renewcommand{\textfraction}{0.1}
\renewcommand{\topfraction}{0.8}
\renewcommand{\floatpagefraction}{0.8}
\setlength{\textfloatsep}{0pt plus 4pt}

\setlength{\textheight}{8 in}
\setlength{\textwidth}{5.8in}
\setlength{\oddsidemargin}{1.2cm}
\setlength{\evensidemargin}{1.2 cm}
\setlength{\marginparwidth}{50pt} 
\setlength{\marginparsep}{5pt} 
\setlength{\topmargin}{1cm}
\setlength{\headheight}{10pt}
\setlength{\headsep}{5pt}
\parskip = 0pt
\parindent = 10pt

\newcommand{\Rrate}[1]{R^{\,#1}}
\newcommand{\Rratek}[2]{\Rrate{#1}_{#2}}
\newcommand{\rparam}[3]{P^{\,#1}_{#2 #3}}
\newcommand{\reaclib}{\mbox{ReacLib}}
\newcommand{\etaone}[2]{\eta^{#1}_{#2}}
\newcommand{\etatwo}[3]{\eta^{#1}_{#2#3}}
\newcommand{\etathree}[4]{\eta^{#1}_{#2#3#4}}
\newcommand{\sideparg}[1]{
\medskip\noindent{\bf\em #1:}
}


\begin{document}

%\begin{frontmatter}

\centerline{\LARGE\bf A Controlled Approximation for Solving Large Kinetic}
\vspace{1ex}
\centerline{\LARGE\bf Networks Coupled to Hydrodynamics}

% \vspace{5pt}
% \centerline
% {\LARGE\bf Explicit Integration for Thermonuclear Networks}

\bigskip
\centerline{Mike Guidry and Nick Brey}
%\ead{guidry@utk.edu}

\centerline{\small\em Department of Physics and Astronomy, University of 
Tennessee}

\centerline{\small\em Knoxville, TN 37996-1200, USA}


%\date{\today}

\vspace{7pt}

\begin{small}
\noindent
These notes propose the use of algebraically stabilized explicit methods to 
give a fast approximate solution of large kinetic networks coupled to 
multi-dimensional radiation hydrodynamics.  These methods exhibit control 
of local error in the integration that permits accuracy to be traded off 
systematically against speed of the network integration.  This leads to 
approximate solutions of the kinetic network that are much faster than with 
standard methods, but with errors that are controlled within acceptable 
tolerance; specifically, with error restricted to be smaller 
than other sources of error coming from the fluid dynamics simulation. Our 
discussion 
will emphasize applications to astrophysical thermonuclear and neutrino 
transport 
networks, but similar methods should be applicable in a number of other 
disciplines that require solution of large kinetic networks coupled to 
multi-dimensional fluid dynamics simulations.
\end{small}


\section{\label{intro} Introduction}

We are presently developing a C++ implementation of the explicit algebraic 
approximation for solving large sets of extremely stiff differential equations 
that was introduced in Refs.\ 
\cite{guidJCP,guidAsy,guidQSS,guidPE,brockJCP,haidar2016}. 
These methods are based on three complementary approaches to solving stiff sets 
of equations 
explicitly: 

\begin{itemize}
 \item 
 the {\em explicit asymptotic algorithm} (Asy),
 \item
 the {\em quasi-steady-state algorithm} (QSS), and
 \item
 the {\em partial equilibrium algorithm} (PE),
\end{itemize}
that stabilize explicit integration through algebraic constaints
and permit much larger timesteps than
normally are possible with explicit methods  (often by many orders of 
magnitude). 
Prior proof-of-principle work has demonstrated that these methods are often 
much 
faster than standard implicit approaches to large sets of stiff equations, with 
the advantage of the explicit methods growing with network size and 
with deployment on limited capability devices like GPUs.
The new C++ code is intended to be a general-purpose community code that can be 
used in a variety astrophysical applications where large sets of stiff ordinary 
differential equations must be integrated that are coupled dynamically to a 
description of the fluid dynamics (typically 3D radiation hydrodynamics) and 
neutrino transport. 

The code is presently being developed for serial 
application, but once that is thorougly debugged and tested we intend to port 
the code to GPUs, since we 
have already demonstrated that these new algorithms work extremely efficiently 
 on GPUs \cite{brockJCP,haidar2016}.  Because the largest present 
petascale and coming exascale machines derive much of their speed from GPU 
accelerators, we expect that this codebase will be particularly useful in 
large-scale astrophysical simulations on those machines. 

The present 
code is aimed at astrophysical applications but the algorithms themselves 
should be 
applicable with relatively small modification for a variety of scientific and 
technological applications where large stiff sets of equations must be solved.
For example, we are deploying them for the solution of neutrino 
transport equations under supernova or neutron star merger conditions 
\cite{endeve20}.


\section{Controlled Versus Uncontrolled Approximations}

Realistic scientific systems are typically strongly coupled multiphysics 
environments for which
all computer simulations must of necessity 
employ approximations to make the problem tractable for a given epoch of 
hardware and software.  Thus, for such problems one has no choice in 
{\em whether} to approximate, but one does have a choice in {\em how} to 
approximate. In this context, one crucial distinction among approximations is 
whether the errors associated with a given approximation controlled or 
uncontrolled.

\begin{itemize}
 \item 
 {\em Controlled approximations}
  mean that the user has 
available parameters that restrict the amount of local error and thus limit the 
net 
global error in the simulation in a quantitative way.
\item
{\em Uncontrolled approximations} mean that the user has available (at 
best) only a qualitative understanding of how much error is introduced by the 
approximation.
\end{itemize}

\noindent
In the specific problem of coupling large thermonuclear networks to 3D 
hydrodynamics in astrophysical simulations of events like supernova explosions 
or neutron star mergers, there is some agreement that a realistic thermonuclear 
network for coupling to the fluid dynamics requires integration of 
extremely stiff equations describing the evolution of $\sim 150$ appropriately 
selected isotopes. Simulations to date have been unable to couple 
networks of that size  systematically to realistic fluid dynamics. Thus in the 
most realistic simulations a severely truncated thermonuclear network has 
been used, often with at most 10-15 isotopes. This is an example of an 
{\em uncontrolled approximation,} for there is no systematic way to estimate 
the resulting error.  Various studies have addressed this problem by trying to 
estimate in some fashion the error introduced by using a small network, but 
these results are at best generic and qualitative, and offer no way to 
determine the error associated with network approximations in any given 
simulation. 

An alternative approach is to use a network of more correct size (say 150 or 
more isotopes), but to gain the speed required not by reducing the size of 
the network, but by solving the realistic network only approximately by 
 a {\em controlled approximation} that permits speed to be traded for 
acceptable accuracy.  


\section{Controlled Algebraically-Stabilized Explicit Solutions}

Explicit methods are generally easier to implement than implicit ones because 
they require only the current state of the system to compute a future state, 
and 
computing an explicit timestep is generally much faster and less memory 
intensive than computing an implicit timestep for large sets of equations. 
However  traditional explicit 
methods are impractical for solving the extremely stiff systems 
characteristic of many astrophysical processes because stability 
requirements restrict the integration timesteps to tiny values, which far 
outweighs the faster explicit computation of each timestep.  Thus explicit 
methods have been thought impractical for stiff systems, 
forcing the use of implicit methods.  Because of the matrix inversions, these 
implicit methods are prohibitively slow for networks of realistic size and 
complexity when coupled to a large-scale fluid-dynamics simulation.  This has 
typically forced the use of drastic approximations employing networks that 
are too small and/or too schematic to be realistic.  This approximation is {\em 
uncontrolled} because it introduces systematic error that is difficult to 
estimate because it is dictated by ability to integrate the network implicitly 
rather than by the realistic physics of the network.

But in a large-scale astrophysical simulation of say a supernova, typically a 
multidimensional hydrodynamical code is coupled to kinetic networks (coupled 
ordinary differential equations) representing thermonuclear burning and/or 
neutrino transport, and it may be argued that these kinetic networks need be 
solved only {\em approximately,} provided that the approximation introduces 
{\em controlled errors} that are  smaller than the errors and 
uncertainties associated with the overall coupled hydrodynamical--kinetic 
system. Given the current state of the art in such simulations, a conservative 
rule of thumb is that (controlled) errors of several percent  or less are 
acceptable for 
approximate kinetic networks, since to do better will not improve the final 
results and thus is a waste of resources. This then raises the question of 
whether there are approximations for standard explicit methods that could allow 
taking more competitive timesteps with an acceptable controlled error, thus 
permitting the solution of larger and more realistic kinetic networks.


Early work coupling our new methods to hydrodynamics in the Flash code.  The 
hydro profiles with "viktor" in the name are from his work \cite{chup2008}:

Chupryna, Viktor, "Explicit Methods in the Nuclear Burning Problem for Supernova 
Ia Models. " PhD diss., University of Tennessee, 2008.
%\url{https://trace.tennessee.edu/utk_graddiss/484}

Early testing of the new explicit methods against Raph's implicit code 
\cite{fege2011}:

Feger, Elisha Don, "Evaluating Explicit Methods for Solving Astrophysical 
Nuclear Reaction Networks. " PhD diss., University of Tennessee, 2011.
%\url{https://trace.tennessee.edu/utk_graddiss/1048}


\clearpage

\setcounter{section}{0}
\appendix

\newpage

\bibliographystyle{unsrt}
\begin{thebibliography} {99}

% % Reactive flows largely in a chemical context
% \bibitem{oran05} E.S. Oran and J.P. Boris, Numerical Simulation of Reactive
% Flow, Cambridge University Press, 2005.
% 
% % Gear book
% \bibitem{gear71}C.W. Gear, Numerical Initial Value Problems in Ordinary
% Differential Equations, Prentice Hall, 1971.
% 
% % Lambert book
% \bibitem{lamb91}J.D. Lambert, Numerical Methods for Ordinary Differential
% Equations, Wiley, 1991.
% 
% % Numerical recipes
% \bibitem{press92}W.H. Press, S.A. Teukolsky, W.T. Vettering, and B.P. Flannery,
% Numerical Recipes in Fortran, Cambridge University Press, 1992.
% 
% % "Thermonuclear Kinetics in Astrophysics"; astro-ph/0509698
% \bibitem{hix05}W.R. Hix and B.S. Meyer, Thermonuclear kinetics in astrophysics,
% Nuc.\ Phys.\ A 777 (2006) 188-207.
% 
% % Comparison of different methods for solving astro networks
% \bibitem{timmes}F.X. Timmes, Integration of Nuclear Reaction Networks for
% Stellar Hydrodynamics, ApJS 124 (1999) 241-263.

% ``Algebraic Stabilization of Explicit Numerical Integration for 
% Extremely Stiff Reaction Networks'' 
\bibitem{guidJCP}Mike Guidry, J. Comp.\ Phys.\ {\bf 231}, 5266-5288 (2012). 
[arXiv:1112.4778].

% ``Explicit Integration of Extremely-Stiff Reaction Networks: 
%  Asymptotic Methods'' 
\bibitem{guidAsy} M. W. Guidry, R. Budiardja, E. Feger, J. J. Billings, W. R. 
Hix, O.
E. B. Messer, K. J. Roche, E. McMahon, and M. He, Comput.\ Sci.\ Disc.\ {\bf 6},
015001 (2013) [arXiv: 1112.4716].

% ``Explicit Integration of Extremely-Stiff Reaction Networks: 
%  Quasi-Steady-State Methods''
\bibitem{guidQSS} M. W. Guidry and J. A. Harris, Comput.\ Sci.\ Disc.\ {\bf 6},
015002 (2013) [arXiv: 1112.4750]

% ``Explicit Integration of Extremely-Stiff Reaction Networks: 
% Partial Equilibrium Methods'' 
\bibitem{guidPE} M. W. Guidry, J. J. Billings, and W. R. Hix, Comput.\ Sci.\ 
Disc.\ {\bf 6}, 015003 (2013) [arXiv: 1112.4738]

% ``Explicit integration with GPU acceleration for large kinetic networks''
\bibitem{brockJCP}Benjamin Brock, Andrew Belt, Jay Jay Billings, and Mike 
Guidry, J. Comp.\ Phys.\ {\bf 302}, 591-602 (2015).

% ``Performance Analysis and Acceleration of Explicit Integration 
% for Large Kinetic Networks using Batched GPU Computations''
\bibitem{haidar2016}A. Haidar, B. Brock, S. Tomov, M. Guidry, J. J. Billings, D. 
Shyles, and J. Dongarra, ``Performance Analysis and Acceleration of Explicit 
Integration for Large Kinetic Networks using Batched GPU Computations'' 2016 
IEEE High Performance Extreme Computing Conference, HPEC (2016).

% Explicit neutrino transport code
\bibitem{endeve20}E. Endeve et al, unpublished.

% % \reaclib\  library
% \bibitem{raus2000}T. Rauscher and F.-K. Thielemann, Astrophysical Reaction Rates
% From Statistical Model Calculations, At.\ Data Nuclear Data Tables 75 (2000)
% 1-351.
% 
% % A QSS Solver for the Stiff Ordinary Differential Equations of Reaction
% % Kinetics
% \bibitem{mott00} D.R. Mott, E.S. Oran, and B. van Leer, Differential Equations
% of Reaction Kinetics, J. Comp.\ Phys.\ 164  (2000) 407-428.
% 
% % Mott thesis
% \bibitem{mott99} D.R. Mott, New Quasi-Steady-State and Partial-Equilibrium
% Methods for Integrating Chemically Reacting Systems, doctoral thesis, University
% of Michigan, 1999.
% 
% % Feger thesis
% \bibitem{feg11b} E. Feger, Evaluating Explicit Methods for Solving Astrophysical
% Nuclear Reaction Networks, doctoral thesis, University of Tennessee, 2011.
% 
% % Mott PE methods paper
% \bibitem{mott03}D. Mott, E. Oran, and B. van Leer, Identifying and Imposing
% Partial Equilibrium in Chemically Reacting Systems, in AIAA-2003-667, 41st
% Aerospace Sciences Meeting and Exhibit, Reno, Nevada, 2003.
% 
% % The Hix-Thielemann network code Xnet
% \bibitem{raphcode}W.R. Hix and F.-K. Thielemann, Computational methods for
% nucleosynthesis and nuclear energy generation, J. Comp.\ Appl.\ Math.\ 109
% (1999) 321-351.
% 
% % JINA reaction rates
% \bibitem{JINA}Tabulated at http://groups.nscl.msu.edu/jina/reaclib/db/. The JINA
% extensions to \reaclib\  are discussed in R.H. Cyburt, et al., The JINA 
% \reaclib\ 
% Database:Its Recent Updates and Impact on Type-I X-ray Bursts, Ap.\ J.\ Supp.\
% 189 (2010) 240-252.

\end{thebibliography}


\end{document}



% Table \ref{tb:reaclibClasses}.
% %
% %
% \begin{table}[t]
%   \centering
%   \caption{\reaclib\ reaction classes  \cite{raus2000} }
%   \label{tb:reaclibClasses}\vspace{0pt}
%       \setlength{\tabcolsep}{10 pt}
%       \begin{tabular}{cll}
%         \hline
%             {Class} &
%             {Reaction} &
%             {Description or example}
% 
%         \\        \hline
%             1 &
%             a $\rightarrow$ b &
%             $\beta$-decay or e$^-$ capture
% 
%         \\ 
%             2 &
%             a $\rightarrow$ b + c &
%             Photodisintegration
%                                  + $\alpha$ 
% 	\\ 
%             3 &
%             a $\rightarrow$ b + c + d&
%             $^{12}$C $\rightarrow$ 3$\alpha$ 
% 	\\ 
%             4 &
%             a + b $\rightarrow$ c &
%             Capture reactions
% 	\\ 
%             5 &
%             a + b $\rightarrow$ c + d &
%             Exchange reactions
% 	\\ 
%             6 &
%             a + b $\rightarrow$ c + d + e &
%             $^2$H + $^7$Be $\rightarrow$ $^1$H + 2$^4$He 
% 	\\ 
%             7 &
%             a + b $\rightarrow$ c + d + e + f &
%             $^3$He + $^7$Be $\rightarrow$ 2$^1$H 
%             + 2$^4$He 
% 	\\ 
%             8 &
%             a + b + c $\rightarrow$ d \, (+ e) &
%             Effective 3-body reactions  
%         \\        
%         \hline
%       \end{tabular}
% \vspace{10pt}
% \end{table}
% %
% %


 
